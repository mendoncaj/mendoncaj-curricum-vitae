%%%%%%
% Jefferson Carlos de Mendonca - curriculum vitae
% Edited using vim and texmaker
% Uses LaTeX and moderncv
%%%%%%
\documentclass[11pt,a4paper]{moderncv}
\moderncvtheme{classic}


\usepackage[utf8]{inputenc}
\usepackage[scale=0.8]{geometry} % adjust the page margins
\usepackage{datenumber,fp}
 
\PassOptionsToPackage{pdftex, 
	colorlinks=true, 
	urlcolor=cyan, 
	linkcolor=cyan, 
	filecolor=cyan, 
	citecolor=cyan}{hyperref}
 
\AtBeginDocument{\setlength{\maketitlenamewidth}{11,5cm}}
\AtBeginDocument{\recomputelengths}

%----------------------------------------------------------------------------------------
%	NAME AND CONTACT INFORMATION SECTION
%----------------------------------------------------------------------------------------

\firstname{\fontsize{25}{30}\selectfont Jefferson}
\familyname{\fontsize{25}{30}\selectfont Mendonça}
\title{Consultor de Software - Especialista Java}

% All information in this block is optional, comment out any lines you don't need
\address{Av. Dr. Orêncio Vidigal, 598 Apto. 51, Bl. 5}{Penha - São Paulo - SP, CEP 03640-010}
\mobile{+55 (11) 97411-4534}
\email{jdev.mendonca@gmail.com}
\linkedin{linkedin.com/in/mendoncaj}

%\homepage{linkedin.com/in/mendoncaj}{linkedin.com/in/mendoncaj}
\quote{"Seja a mudança que você deseja para o mundo" - Mahatma Gandhi}   


% --------------------------------------
% CONTENT
% --------------------------------------

\begin{document}

\maketitle

\section{Sobre}
\cvcomputer
{\textbf{Idade}}{31 anos}
{\textbf{Estado civil}}{Casado}
\cvcomputer
{\textbf{Fumante}}{Não}
{\textbf{Filhos}}{Não}

\section{Formação Acadêmica}
\cventry{Setembro 2016 -- Setembro 2019}{Escola Politécnica da Universidade de São Paulo - USP}{}{Mestrado em Engenharia da Computação}{}{}

\cventry{Janeiro 2006 -- Dezembro 2010}{Universidade Presbiteriana MACKENZIE}{}{Ciências da Computação}{}{}

\section{Certificação}
\cventry{Agosto - 2013}{OCJP}{Score 88\%}{Oracle Certified Java Programmer}{}{}

\section{Experiência Profissional}

\cventry{Março 2017 -- Atualmente}{Consultor de Software}{}{\textbf{Cardif}}{}{\textbf{Desenvolvimento de micro serviços com TiBCO.} Atuo na evolução das aplicações de sinistros, seguros, faturas, recebimento entre outros. Aplicações que utilizam Java, ActiveMQ, Hibernate e Spring Framework como principais tecnologias. Componho o time responsável pelo desenvolvimento e governança de serviços utilizando a solução de \textbf{ESB - Business Works TiBCO}. Nesse projeto estamos desenvolvendo um \textbf{catálogo de serviços} para abertura de sinistros, consulta de apólices, terceiros e demais funcionalidades que serão expostos para as aplicações mobile do grupo na América Latina. (México, Panamá, Colômbia, Chile, Peru e Brasil). 
\\ \\ Atuei no desenvolvimento e manutenção de aplicação para carga massiva de dados utilizando \textbf{Spring Batch/Spring Integration} e Jobs agendados para processamento de ciclos de vida de apólices. Refatoração de código, testes e correções nos sistemas internos e ministrei treinamentos sobre padrões de projetos e qualidade de software para o time de desenvolvimento da Cardif no México.} \\


\cventry{Agosto 2015 -- Fevereiro 2017}{Líder Técnico}{}{\textbf{Editora Globo S.A}}{}{\textbf{Arquitetura e Modelagem dos novos aplicativos da Editora Globo.} Os aplicativos foram desenvolvidos utilizando a plataforma AEM-M da Adobe. O grande desafio deste projeto foi integrar o CMS de publicação de conteúdo (matérias jornalí­sticas) com a plataforma da Adobe. Um de nossos aplicativos foi desenvolvido com o Infoglobo - outra empresa do \textbf{Grupo Globo}. Atuei na arquitetura da solução para garantir as integrações entre os diversos sistemas das empresas. Nesse projeto também atuei com a plataforma de autenticação da \textbf{globo.com} chamada de GloboID a mesma utilizada pelos serviços \textbf{Globo Play} e \textbf{Cartola}. Dentre as diversas tecnologias utilizadas destaco o controle de mensagens (Produtor vs Consumidor) utilizando filas de gerenciamento no \textbf{Rabbit MQ}, Log Centralizado \textbf{MongoDB} e WebServices (Soap e Rest). 
\newline \newline Trabalhei na aplicação em \textbf{NodeJS} com \textbf{Mongoose} para propor alternativas de editores de texto web, com o intuito de os jornalistas terem maior agilidade e flexibilidade na composição de matérias jornalísticas e publicitárias.} \\


\cventry{Junho 2013 -- Agosto 2015}{Desenvolvedor WEB III}{}{\textbf{Editora Globo S.A}}{}{\textbf{Desenvolvimento de um Middleware para autorização às plataformas digitais.} Integração entre o sistema de assinaturas da Editora Globo com o Adobe Digital Publishing e o E-commerce da empresa. Projeto web desenvolvido em Java EE 6, com endpoints implementados em Rest JAX-RS e SOAP JAX-WS, injeção de dependência (CDI Weld) e  acesso a dados com JPA. Além das tecnologias citadas, utilizamos DynamoDB, JUnit, Mockito e Eclemma (code coverage) com estimativas baseadas em Metodologias Ágeis tais como Scrum e Kanban.
\\ \\ Atuei no desenvolvimento de um Hot site para venda de assinaturas digitais das revistas da Editora Globo. Sendo a aamada web desenvolvida em Spring MVC e engine de templates em Apache Tiles, com integrações via Web Services com o e-commerce e o sistema de assinturas da empresa} \\


\cventry{Maio 2011 -- Maio 2013}{Desenvolvedor WEB II}{}{Editora Globo S.A}{}{\textbf{Mundo do Sítio, ambiente virtual com atividades e jogos e educativos.} Colaborei no desenvolvimento do "Cantinho dos Pais" - Área restrita onde os pais podiam gerenciar suas assinaturas e visualizar as informações gerada por seus filhos, como por exemplos os jogos mais acessados, tempo de atividades, mensagens trocadas etc. Para ilustrar estas informaçõees, tabelas e gráficos eram disponibilizados. O desenvolvimento desse site administrativo foi feito em JSF 1.2 e 2.0 com RichFaces 3.3 e 4.0.} \\


\cventry{Fevereiro 2009 -- Abril 2011}{Analista Programador}{}{Editora Globo S.A}{}{Auxí­lio nas etapas de desenvolvimento e testes dos sites: Revista Época, Revista Auto Esporte, Revista Crescer etc. Criação de páginas web utilizando JSP e Servlet 2.5 e transformação de dados para envio de email utilizando XSTL. 
\\ \\ Trabalhei em correções pontuais, manutenção e restruturação do site "Livros para uma cuca bacana" (desenvolvido em ASP clássico). Criação de Quiz e Concursos Culturais utilizando Html, JS e Jquery.} \\

% --------------------------------------
% TRAVELS
% --------------------------------------
\section{Viagens}

\cvitem{México}{Março de 2018 - Instalação e estabilização de novas versões de softwares e realização de treinamentos sobre Modelo de Dados, Padrões de Projetos, Orientação a Objetos e Qualidade de Software. \textbf{Cardif - Cidade do México}.}

\cvitem{México}{Janeiro de 2018 - Desenvolvimento e correção de sistemas (plugins desenvolvidos no país), treinamento sobre testes unitários e de integração. \textbf{Cardif - Cidade do México}.}

\cvitem{Estados Unidos}{Abril de 2016 - Adobe Summit e reunião de alinhamento para o projeto AEM-Mobile (DPS) em \textbf{Las Vegas - Nevada}.}

\cvitem{Estados Unidos}{Setembro de 2015 - Reunião de alinhamento para o projeto New DPS 2015 (Mobileauth) na sede da Adobe Systems, com desenvolvimento para consumo de API's e testes com os engenheiros da Adobe, localizado em \textbf{São Francisco - Califórnia}.}

\cvitem{Canadá}{Agosto de 2014 - Intercambio de verão em \textbf{Vancouver - British Columbia}.}


% --------------------------------------
% Courses
% --------------------------------------

\section{Cursos e Especializações}

\cventry{Junho 2017}{Integration Patterns com Apache Camel e Spring Integration}{}{Elder da Rocha}{}{Duração 40 horas}
\cventry{Outubro 2015}{Arquiteto de Software}{}{Global Code}{}{Duração 120 horas}
\cventry{Agosto 2015}{WSO2 ESB for Developers - Fundamentals}{}{WSO2}{}{Duração 40 horas}
\cventry{Julho 2015}{JDK 8 Lambdas and Streams}{}{Oracle, Massive Open Online Course}{}{Duração 3 semanas}
\cventry{Março 2015}{Architecting on AWS}{}{Amazon}{}{Duração 16 horas}
\cventry{Dezembro 2013}{M101J: MongoDB for Java Developers}{}{Mongo University Online Course}{}{Duração 7 semanas}
\cventry{Dezembro 2010}{Python para Profissionais}{}{Luciano Ramalho}{}{Duração 40 horas}

\section{Idiomas}
\cvitem{Português}{Nativo}
\cvitem{Inglês}{Avançado}
\cvitem{Espanhol}{Básico}


\section{Habilidades Técnicas}
\cvcomputer
{\textbf{Java Essentials}}{EJB, JSP, JSTL, JSF, Facelets, JAVA EE, Spring MVC, CDI, JAVA SE, JPA/Hibernate/EhCache, JDBC, JOOQ, XSLT}
{\textbf{Programming Languages}}{Java, C/C++, Python, NodeJS, ASP Clássico}
\cvcomputer
{\textbf{Web Services}}{Soap - Jax-WS, Rest Jax-RS}
{\textbf{Miscellaneous}}{Scrum, Kanban, Sonar Qube, JMS - Rabbit MQ}
\cvcomputer
{\textbf{Template Engine}}{Apache Tiles, Freemarker, Velocity}
{\textbf{Data Parser}}{Jackson JSON, JAXB, JDOM, XmlBeans, Xstream, Apache POI - Excel Framework}
\cvcomputer
{\textbf{Test}}{JUnit, Mockito, Selenium, JMeter, Cobertura, Code Coverage, FindBugs}
{\textbf{Build Tools}}{Ant, Maven, Gradle}
\cvcomputer
{\textbf{Application Server}}{Apache Tomcat, Glassfish, JBoss, 4.5 e 7.1, Websphere}
{\textbf{Amazon AWS}}{Elastic BeansTalk, EC2, RDS, Pipeline, DynamoDB}
\cvcomputer
{\textbf{Database}}{Mysql, Oracle, MongoDB}
{\textbf{Modeling and SOA}}{UML, ER - Entidade Relacionamento, ESB TiBCO, WSO2}
\cvcomputer
{\textbf{Code Management}}{Git, SVN - SubVersion}
{\textbf{Front End}}{JavaScript, JQuery, Angular.js, HTML/CSS}
{}{}

% --------------------------------------
% INTERESTS HOBBIES
% --------------------------------------
\section{Interesses e Hobbies}

\cvitem{Indispensável}{Estar com a minha esposa, família e amigos, de preferência vendo o mar.}
\cvitem{Necessário}{Conhecer novas tecnologias, praticar esportes - corrida e natação.}
\cvitem{Descontrair}{Assistir séries, documentários e NFL, além de jogar vídeo game.}
\cvitem{Prazer}{Tocar bateria, ouvir Rock n' Roll, AC/DC, Rolling Stones, Ozzy Osbourne, Dio e outros.}
% --------------------------------------
\end{document}