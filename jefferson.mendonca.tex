
%%%%%%
% Jefferson Carlos de Mendonca - curriculum vitae
% Edited using vim and texmaker
% Uses LaTeX and moderncv
%%%%%%
\documentclass[11pt,a4paper]{moderncv}
\moderncvtheme[cyan]{classic}

\usepackage[utf8]{inputenc} % character encoding
\usepackage[T1]{fontenc}
\usepackage[scale=0.8]{geometry} % adjust the page margins
\usepackage{datenumber,fp}
 
\PassOptionsToPackage{pdftex, 
	colorlinks=true, 
	urlcolor=cyan, 
	linkcolor=cyan, 
	filecolor=cyan, 
	citecolor=cyan}{hyperref}
 
\AtBeginDocument{\setlength{\maketitlenamewidth}{11,5cm}}
\AtBeginDocument{\recomputelengths}

% personal data
\firstname{\fontsize{25}{30}\selectfont Jefferson Carlos}
\familyname{\fontsize{25}{30}\selectfont de Mendonça}
%% \title{Analista Desenvolvedor Java}
\title{Analista Desenvolvedor Java}

\address{Av. Dr. Orêncio Vidigal, 598 Apto. 51, Bl. 5}{Penha - São Paulo, SP \\ CEP 03640-010}
\phone{+55 (11) \hspace{1 mm}  2894-1051 }
\mobile{+55 (11) 97411-4534}
\email{jdev.mendonca@gmail.com}
%\github{github.com/mendoncaj}
\linkedin{linkedin.com/in/mendoncaj}
% \photo[64pt]{picture}                         % '64pt' is the height the picture must be resized to and 'picture' is the name of the picture file;

\quote{"Seja a mudança que você deseja para o mundo" Mahatma Gandhi}       


% --------------------------------------
% CONTENT
% --------------------------------------

\begin{document}
\maketitle

\section{Sobre}
\cvcomputer
{\textbf{Idade}}{30 anos}
{\textbf{Estado civil}}{Casado}
\cvcomputer
{\textbf{Fumante}}{Não}
{\textbf{Filhos}}{Não}

\section{Formaç\~{a}o Acad\^{e}mica}
\cventry{Setembro 2016}{Escola Politécnica da Universidade de São Paulo - USP}{}{Mestrado em Engenharia da Computação}{}{}

\cventry{Janeiro 2006 -- Dezembro 2010}{Universidade Presbiteriana MACKENZIE}{}{Ciências da Computação}{}{}

\section{Certificaç\~{a}o}
\cventry{Agosto - 2013}{OCJP}{Score 88\%}{Oracle Certified Java Programmer}{}{}

\section{Experiência Profissional}

\cventry{Abril 2014 -- Atualmente}{Desenvolvedor WEB IV}{}{\textbf{Editora Globo S.A}}{}{\textbf{Middleware Mobileauth para Autorização de Conteúdo Digital.} Integração entre o sistema de assinaturas da Editora Globo com o Adobe Digital Publishing e E-commerce da empresa. Projeto web desenvolvido em Java EE 6, com endpoints implementados em Rest JAX-RS e SOAP JAX-WS, injeção de dependência (CDI Weld) e  acesso a dados com JPA. Além das tecnologias citadas, utilizamos DynamoDB, JUnit, Mockito e Eclemma (code coverage) com estimativas baseadas em metodologias ágeis tais como XP e Scrum.}

\cventry{Junho 2013 -- Março 2014}{Desenvolvedor WEB III}{}{Editora Globo S.A}{}{\textbf{GloboMagazines.} Hot site para venda de assinaturas digitais das revistas da Editora Globo. Camada web desenvolvida em Spring MVC, na validação do lado do servidor utilizamos Beans Validation, nossa engine de templates está em Apache Tiles e no front end utilizamos Javascript com Jquery. As Regras de Negócio estão distribuídas no e-commerce e no projeto Mobileauth as integrações com esses sistemas foram implementadas com SOAP JAX-WS e Rest JAX-RS. Para build, e deploy e gerência de dependências utilizamos Maven e para controle de versões Git.}

\cventry{Maio 2011 -- Maio 2013}{Desenvolvedor WEB I}{}{Editora Globo S.A}{}{\textbf{Mundo do Sítio, ambiente virtual com atividades e jogos e educativos.} Colaborei no desenvolvimento do "Cantinho dos Pais" - Área restrita onde os pais podiam gerenciar suas assinaturas e visualizar as informações gerada por seus filhos, como por exemplos os jogos mais acessados, tempo de atividades, mensagens trocadas etc. Para ilustrar estas informações, tabelas e gráficos eram disponibilizados. O desenvolvimento desse site administrativo foi feito em JSF 1.2 e 2.0 com RichFaces 3.3 e 4.0.}

\cventry{Outubro 2009 -- Abril 2011}{Analista Programador}{}{Editora Globo S.A}{}{Auxílio nas etapas de desenvolvimento e testes de sites web, Revista Época, Revista Auto Esporte, Revista Crescer. Criação de páginas web utilizando JSP e Servlet 2.5 e transformação de dados para envio de email utilizando XSTL.}

\cventry{Fevereiro 2009 -- Setembro 2009}{Estagiário}{Desenvolvedor de Sistemas}{Editora Globo S.A}{}{Correções pontuais em sites web, manutenção e restruturação do site "Livros para uma cuca bacana" (desenvolvido em ASP clássico). Criação de Quiz e Concursos Culturais utilizando Html, JS e Jquery.}

\cventry{Fevereiro 2008 -- Janeiro 2009}{Estagiário}{Analista de Segurança Digital}{\textbf{Nextel Telecomunicações Ltda}}{}{Atender solicitações de acesso a sistemas, diretório de rede, caixas postais e listas de distribuições. Auxiliar na elaboração e documentação de programas. Avaliação de novos sistemas administrativos e corporativos, testar novas fases e melhorias de sistemas.}

\cventry{Fevereiro 2004 -- Dezembro 2005}{Estagiário}{Assistente Administrativo}{\textbf{Banco do Brasil S.A}}{}{Monitoramento dos terminais de Auto Atendimento (AAT), para prevenção ou correção de falhas através de sistemas de controle online. Elaboração de manuais dos softwares e aplicações utilizadas. Treinamento de novos estagiários contratados.}

\section{Habilidades Técnicas}
\cvcomputer
{\textbf{Java Essentials}}{EJB, JSP, JSTL, JSF, Facelets, JAVA EE, Spring MVC, CDI, JAVA SE, JPA/Hibernate/EhCache, JDBC, JOOQ, Java Mail, XSLT, Log4J}
{\textbf{Linguagens}}{Java, C/C++, Python, ASP Clássico, JavaScript, HTML/CSS}
\cvcomputer
{\textbf{Web Services}}{Soap - Jax-WS, Rest Jax-RS}
{\textbf{Integração Contínua}}{Jenkins, Sonar Qube}
\cvcomputer
{\textbf{Source Code Management}}{Git, SVN - SubVersion}
{\textbf{Data Parser}}{Jackson JSON, JAXB, JDOM, XmlBeans, Xstream, Apache POI - Excel Framework}
\cvcomputer
{\textbf{Testes}}{JUnit, Mockito, Selenium, JMeter, Cobertura, Code Coverage, FindBugs}
{\textbf{Build Tools}}{Ant, Maven, Gradle}
\cvcomputer
{\textbf{Amazon AWS}}{Elastic BeansTalk, EC2, RDS, Pipeline, DynamoDB}
{\textbf{Application Server}}{Apache Tomcat, JBoss, 4.5 e 7.1}
\cvcomputer
{\textbf{Metodologia}}{XP, Scrum}
{\textbf{Client Side}}{JavaScript, JQuery, Angular.js}
\cvcomputer
{\textbf{Big Data}}{Apache Spark, Java 8 Streams \& Lambda}
{\textbf{Template Engine}}{Apache Tiles, Freemarker, Velocity}
\cvcomputer
{\textbf{Database}}{Mysql, Oracle}
{\textbf{Modelagem e SOA}}{UML, ER - Entidade Relacionamento, ESB WSO2}
\cvcomputer
{\textbf{OS's}}{Linux, Mac, Windows}
{}{}


\section{Cursos e Especializaç\~{o}es}

\cventry{Outubro 2015}{Arquiteto de Software}{}{Global Code}{}{Duração 120 horas}
\cventry{Agosto 2015}{WSO2 ESB for Developers - Fundamentals}{}{WSO2}{}{Duração 40 horas}
\cventry{Julho 2015}{JDK 8 Lambdas and Streams}{}{Oracle, Massive Open Online Course}{}{Duração 3 semanas}
\cventry{Março 2015}{Architecting on AWS}{}{Amazon}{}{Duração 16 horas}
\cventry{Dezembro 2013}{M101J: MongoDB for Java Developers}{}{Mongo University Online Course}{}{Duração 7 semanas}
\cventry{Dezembro 2010}{Python para Profissionais}{}{Luciano Ramalho}{}{Duração 40 horas}

\section{Idiomas}
\cvitem{Português}{Nativo}
\cvitem{Inglês}{Avançado, FCE First Certificate in English. \textbf{Vancouver, Canadá}}

% --------------------------------------
% TRAVELS
% --------------------------------------
\section{Viagens}

\cvitem{Estados Unidos}{Abril de 2016 - Adobe Summit e reunião de alinhamento para o projeto AEM-Mobile (DPS) em \textbf{Las Vegas - Nevada}.}

\cvitem{Estados Unidos}{Setembro de 2015 - Reunião de alinhamento para o projeto New DPS 2015 (Mobileauth) na sede da Adobe Systems, localizado em \textbf{São Francisco - Califórnia}.}

\cvitem{Canadá}{Agosto de 2014 - Intercâmbio de verão em \textbf{Vancouver - British Columbia}.}

% --------------------------------------
% INTERESTS HOBBIES
% --------------------------------------
\section{Interesses e Hobbies}

\cvitem{Indispensável}{Estar com a minha esposa, fam\'{i}lia e amigos, de preferência vendo o mar.}
\cvitem{Descontrair}{Assistir séries e documentários, jogar video game.}
\cvitem{Prazer}{Conhecer novas técnologias, ouvir Rock n' Roll, ACDC, Rolling Stones, Ozzy Osbourne, Dio}
\cvitem{Necessário}{Praticar esportes, corrida, natação e tocar bateria.}
% --------------------------------------
\end{document}